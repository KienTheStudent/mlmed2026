\documentclass[conference]{IEEEtran}

\usepackage[utf8]{inputenc}
\usepackage{booktabs}

\title{ML Medicine: Labwork Report}
\author{Ngo Xuan Kien}

\begin{document}

\maketitle

\section{Labwork 1: ECG Heartbeat}

\subsection{Introduction}

The dataset utilized is the \textbf{ECG Heartbeat Categorization Dataset}, a collection of heartbeat signals sampled at 125Hz. The study combines four CSV files, resulting in a total of 123,998 samples. Each sample contains 188 columns: 187 time-series features and one target label.

\subsection{Preprocessing and Methodology}

To ensure a robust baseline, the following steps were taken:

\begin{itemize}
    \item \textbf{Data Partitioning:} A \texttt{train\_test\_split} and \texttt{stratify} were used to maintain class distribution between sets.
    \item \textbf{Normalization:} We applied \texttt{StandardScaler} so the Logistic Regression model treats all 187 time-step features with equal weight.
    \item \textbf{Model Configuration:} Logistic Regression was implemented with \texttt{max\_iter=1000} to allow model to find optimal weight.
\end{itemize}

\subsection{Results}

The model achieved a total accuracy of \textbf{84.1\%} but performance varied significantly across the five classes (0--4).

\begin{table}[h]
\centering
\caption{Model Performance per Class}
\begin{tabular}{lcc}
\toprule
\textbf{Class} & \textbf{Recall} & \textbf{F1-Score} \\ \midrule
Class 0 (Normal) & 0.97 & 0.91 \\
Class 1 & 0.19 & 0.29 \\
Class 2 & 0.34 & 0.43 \\
Class 3 & 0.29 & 0.39 \\
Class 4 & 0.87 & 0.90 \\ \bottomrule
\end{tabular}
\end{table}

\subsection{Discussion}

While Classes 0 and 4 show high F1-scores (0.91 and 0.90 respectively), Classes 1, 2, and 3 exhibit low recall (0.19, 0.34 and 0.29 respectively). This indicates that the model frequently misses specific abnormal heartbeats, a direct result of significant class imbalance within the dataset.


\section{Labwork 2: Ultrasound}

\subsection{Introduction}

This labwork introduces a second project using an \textbf{ultrasound-derived tabular dataset} of pre-extracted image features. The prediction target is the image \emph{pixel size (mm)} for each sample.

\subsection{Preprocessing and Methodology}

\begin{itemize}
    \item \textbf{Features and Target:} Numerical image-level features were used as predictors; the target variable is \texttt{pixel size (mm)}.
    \item \textbf{Data Split:} The dataset was partitioned using an 80/20 train--test split
    \item \textbf{Modeling Approach:} LinearRegression is applied
    \item \textbf{Evaluation:} Mean Absolute Error (MAE) was selected
\end{itemize}

\subsection{Results}

\begin{table}[h]
\centering
\caption{Ultrasound — Outlier Summary}
\begin{tabular}{lc}
\toprule
\textbf{Measure} & \textbf{Value} \\ \midrule
Number of outliers & 63 \\

MAE & 0.02 \\ \bottomrule
\end{tabular}
\end{table}

\subsection{Discussion}

Labwork 2 uses a linear regression model on ultrasound-derived features to predict pixel size in millimeters. The model achieves a low \textbf{MAE} of \textbf{0.02}, indicating that the predicted values are very close to the true measurements. However, \textbf{63} detected outliers highlights the importance of careful data preprocessing, as linear models can be sensitive to abnormal samples

\end{document}
